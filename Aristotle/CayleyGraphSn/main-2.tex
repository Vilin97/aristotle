\documentclass{article}

\usepackage[T1]{fontenc}
\usepackage[utf8]{inputenc}
\usepackage[english]{babel}
\usepackage[tbtags]{amsmath}
\usepackage{amsfonts,amssymb}
\usepackage{amsthm}

\usepackage{graphicx}

\usepackage{color}
\usepackage{graphics}
\usepackage[hidelinks]{hyperref}

% Custom commands
\newcommand{\T}[1]{\textup{T}_#1}
\newcommand{\F}[1]{\textup{F}_#1}

\newcommand{\fjt}{\left\lfloor\frac{j}{2}\right\rfloor}
\newcommand{\cjt}{\left\lceil\frac{j}{2}\right\rceil}

\newcommand{\fit}{\left\lfloor\frac{i}{2}\right\rfloor}
\newcommand{\cit}{\left\lceil\frac{i}{2}\right\rceil}
\newcommand{\fimt}{\left\lfloor\frac{i-1}{2}\right\rfloor}
\newcommand{\cimt}{\left\lceil\frac{i-1}{2}\right\rceil}
\newcommand{\fmt}{\left\lfloor\frac{m}{2}\right\rfloor}
\newcommand{\cmt}{\left\lceil\frac{m}{2}\right\rceil}
\newcommand{\fnt}{\left\lfloor\frac{n}{2}\right\rfloor}
\newcommand{\cnt}{\left\lceil\frac{n}{2}\right\rceil}

\newcommand{\fif}{\left\lfloor\frac{i}{4}\right\rfloor}
\newcommand{\fiet}{\left\lfloor\frac{i_e}{2}\right\rfloor}
\newcommand{\fietvar}{\left\lfloor\frac{ie}{2}\right\rfloor}
\newcommand{\cif}{\left\lceil\frac{i}{4}\right\rceil}
\newcommand{\fimf}{\left\lfloor\frac{i-1}{4}\right\rfloor}
\newcommand{\cimf}{\left\lceil\frac{i-1}{4}\right\rceil}
\newcommand{\fmf}{\left\lfloor\frac{m}{4}\right\rfloor}
\newcommand{\cmf}{\left\lceil\frac{m}{4}\right\rceil}
\newcommand{\fnf}{\left\lfloor\frac{n}{4}\right\rfloor}
\newcommand{\cnf}{\left\lceil\frac{n}{4}\right\rceil}

\newcommand{\fcntt}{\left\lfloor\frac{\left\lceil\frac{n}{2}\right\rceil}{2}\right\rfloor}

\newcommand{\ffntt}{\left\lfloor\frac{\left\lfloor\frac{n}{2}\right\rfloor}{2}\right\rfloor}

\newcommand{\cfntt}{\left\lceil\frac{\left\lfloor\frac{n}{2}\right\rfloor}{2}\right\rceil}

\newcommand{\ccntt}{\left\lceil\frac{\left\lceil\frac{n}{2}\right\rceil}{2}\right\rceil}

\newcommand{\n}[1]{n\!-\!#1}
\newcommand{\p}[2]{(\n{#1}\ \ \n{#2})}
\newcommand{\pn}[1]{(n\ \ \n{#1})}

% roman numbers upper-case
\newcommand{\RomanNumeralCaps}[1]
    {\MakeUppercase{\romannumeral #1}}

% Page layout
\usepackage[margin=1in]{geometry}

\let\rom=\textup

\numberwithin{equation}{section}

\theoremstyle{plain}
\newtheorem{theorem}{Theorem}
\newtheorem{lemma}{Lemma}

\begin{document}

\title{A Lower Bound for the Diameter of Cayley Graph of the Symmetric Group $S_n$ Generated by  $(12), (12 \dots n), (1n \dots 2)$}

\author{G.\,V.~Antiufeev\\
\small Independent scientist\\
\small \texttt{grigoriy.rus@gmail.com}}

\date{}

\maketitle

\begin{abstract}
Let us denote elements of the symmetric group $S_n$ using square brackets for one-line notation. Cycles will be represented using parentheses, following standard notation.
Under this convention, the full reversal of the identity element $()$ is the element $s = [n\ n-1 \dots 1]$.

In this article, we obtain a lower bound on the decomposition complexity of elements $s(1n \dots 2)^{i}$ into the generators $(12), (12 \dots n), (1n \dots 2)$, where $i$ ranges over the set $\{1,2,\dots,n\}$.
As a consequence, we derive the lower bound $n(n-1)/2$ for the diameter of Cayley graph of the group $S_n$ generated by $(12), (12 \dots n), (1n \dots 2)$.
\end{abstract}

\noindent\textbf{Keywords:} group theory, graph diameter, Cayley graph, cyclic shift, symmetric group.

\section{Introduction}

One way to define a group is by using a generating set and relations. This approach is often more convenient than, for example, describing the group via its Cayley table.
The structure of a group can be encoded by its Cayley graph, whose vertices correspond to the elements of the group. By multiplying each element by the generators, we obtain vertices connected by an edge to the original element.

In the classic calculation of the diameter, all edges have weight equal to one.
In this case, the diameter of the graph equals the maximum length of the shortest word representing a group element in terms of the given generators and their inverses.
Thus, the diameter reflects the worst-case algorithmic complexity of decomposing a group element into a product of the generators and their inverses.

The symmetric group $S_n$ is of particular interest since, according to Cayley's theorem, any finite group is isomorphic to a subgroup of some symmetric group.
The elements of $S_n$ are permutations, which connects the study of Cayley graphs of this group with sorting problems \cite{Alon}.

One of the minimal generating sets for $S_n$ consists of the two elements $(12)$ and $(12 \dots n)$. This set is also important because the Cayley graph for these generators contains a Hamiltonian path \cite{SawadaWilliams19}, which permits the construction of a Gray code.
Different weights for the edges corresponding to the generators affect the graph's diameter.
For instance, if the weight of all edges corresponding to the element $(1n \dots 2)$ is set to $\infty$, the asymptotic diameter is $\frac{3n^2}{4}$ \cite{Zubov}.
If the weight of all edges corresponding to $(12 \dots n)$ and $(1n \dots 2)$ are set to zero, the diameter equals $\left\lfloor \frac{(n-1)^2}{4} \right\rfloor$, as shown in \cite{Alon}.
In the present work, we assume that all edges of the graph have weight equal to one.
For such a graph, the bounds obtained in \cite{ChervovRL} are: the lower bound is equal to $\frac{n(n-1)}{2}-\frac{n}{2}-1$ and the upper bound is equal to $\frac{n(n-1)}{2}+3n$.

We denote elements of $S_n$ using square brackets for one-line notation.
Cycle notation is represented using parentheses.
Thus, the full reversal of the identity element $()$ of $S_n$ yields the permutation $s = [n\ n-1 \dots 1]$.
In \cite{ChervovRL}, it was hypothesized that the most difficult element, for which the lower bound of the diameter of the Cayley graph of the symmetric group $S_n$ with generators $(12), (12 \dots n), (1n \dots 2)$, equal to $\frac{n(n-1)}{2}$, is attained, is the permutation $s$ shifted by two, i.e., the element $s(1n \dots 2)^{2}$.

In this paper we obtain a lower bound on the decomposition complexity of elements $s(1n \dots 2)^{i}$ into the generators $(12), (12 \dots n), (1n \dots 2)$, where $i$ ranges over the set $\{1,2,\dots,n\}$.
As a consequence, we derive the lower bound $\frac{n(n-1)}{2}$ for the diameter of Cayley graph of the group $S_n$ generated by $(12), (12 \dots n), (1n \dots 2)$.



\section{Definitions}
\label{sec::def}

We introduce several definitions. Additional definitions can be found, for example, in~ \cite{Konstantinova}.

\textit{A permutation} of length $n$ of elements of the set $\mathbb{Z}_n$ (with $n$ used as a representative of $\overline{0}$) is given by:
$$
\pi=
\bigl(\begin{smallmatrix}
  1 & 2 & 3 & \dots & n-1 & n \\
  \sigma(1) & \sigma(2) & \sigma(3) & \dots & \sigma(n-1) & \sigma(n)
\end{smallmatrix}\bigr),
$$
where $\sigma(i)$ denotes the image of the element $i$, $\sigma(i),i \in \mathbb{Z}_n$.
We also use a more compact notation for such permutations, employing square brackets in order to avoid confusion with cycle notation:
$$
\pi =
[\ \pi_1\ \ \pi_2\ \ \pi_3\ \ \dots\ \ \pi_{n-1}\ \ \pi_n\ ]
=
[\ \sigma(1)\ \ \sigma(2)\ \ \sigma(3)\ \ \dots\ \ \sigma(n\!-\!1)\ \ \sigma(n)\ ].
$$
In this notation, unlike the classical two-line representation, the order of the entries is essential.

Let $K$ be a set equipped with a strict linear order $\prec$ (we will omit the word ``strict'' in the following).
Elements $a,b,c \in K$ are said to be in the relation of a \textit{cyclic order $\prec_{cycle}$} if and only if one of the following conditions holds \cite[p.6]{Hunt1935}:
$$
a \prec b \prec c, \quad b \prec c \prec a, \quad c \prec a \prec b.
$$
In this case, we say that the linear order $\prec$ \textit{induces the cyclic order $\prec_{cycle}$}.
An element $z \in K$ \textit{determines a linear order $\prec_z$} on $K \setminus \{z\}$ as follows: $a \prec_z b$ if and only if the elements $a,b,z$ are in the cyclic order~$\prec_{cycle}$ \cite[p.7]{Hunt1935}.
Below, after the introduction of the notion of orbits, an important remark on the preservation of cyclic order will be given.



\textit{The group of all permutations of length
$n$} is denoted by $S_n$, and \textit{the identity element} of the group $S_n$ is denoted by $()$.

\textit{The dihedral group} $
D_n \;=\; \langle R,S \mid R^n = e,\; S^2 = e,\; SRS = R^{-1} \rangle.$ The group $D_n$ has order $2n$.
The identity element of $D_n$ is $e$.

By Cayley's theorem, any group is isomorphic to a subgroup of a symmetric group.
Let $D_n \cong H_n < S_n$.
We construct the subgroup $H_n$ as follows. Let
$r = [2\ 3\ \dots\ n\ 1]$, $s = [n\ n-1 \dots\ 1]$.
The permutation $r$ is called \textit{a left shift}, and the permutation $r^{-1}$ is called \textit{a right shift}.
Then $H_n \;=\; \langle r,s \mid r^n = (),\; s^2 = (),\; srs = r^{-1} \rangle$ and the isomorphism is given by the mapping $\varphi: D_n \longrightarrow H_n$, defined by:
$$\varphi(R^i)=r^i,\quad
\varphi(SR^i)=sr^i,\quad
\varphi^{-1}(r^i)=R^i,\quad
\varphi^{-1}(sr^i)=SR^i.
$$
Thus, we will work with elements of the subgroup $H_n$ as with elements of the dihedral group.

Let the cycle $(12)$ be denoted by $\delta$, $\delta \in S_n$. That is, $\delta = [2\ 1\ 3\ 4\ \dots\ n\!-\!1\ n].$

We consider the Cayley graph $\Gamma$ of the group $S_n$ with generators $\delta,r,r^{-1}$.

\textit{The distance} $\textup{dist}(\pi',\pi'')$ between two permutations $\pi'$ and $\pi''$, where $\pi',\pi'' \in S_n$ is defined as the length of the shortest path in $\Gamma$ connecting the vertices corresponding to $\pi'$ and $\pi''$.
The pair $S_n$ and $\textup{dist}:S_n \times S_n \rightarrow \mathbb{R}$ forms a metric space.

The greatest distance between any two permutations is \textit{the diameter} of $\Gamma$:
$$
\textup{diam}(\Gamma) =
\max_{
\pi,\xi \in S_n
}
\textup{dist}(\pi,\xi).
$$

\textit{The orbit} of the action of the group $\langle r \rangle$ on a permutation $\pi \in S_n$: $\textup{Orb}(\pi) = \{\pi g \mid g \in \langle r \rangle\}$. Clearly, if the elements of the permutation $\pi$ are in a cyclic order, then the elements of any other permutation in $\textup{Orb}(\pi)$ preserve this cyclic order and are in the same cyclic order relation.


Let $\pi \in S_n$ and $\xi \in$~$S_n$.
\textit{The distance} between a permutation
$\pi$ and $\textup{Orb}(\xi)$ is defined as:
$$
\textup{dist}(\pi,\textup{Orb}(\xi)) =
\inf_{
\substack{\xi' \in \textup{Orb}(\xi)}
}
\textup{dist}(\pi,\xi').
$$

A cyclic order can be introduced on the elements of a permutation $\pi \in S_n$, induced by the indices of the elements. As mentioned above, the elements of any other permutation in $\textup{Orb}(\pi)$ preserve this cyclic order. Clearly, multiplying the permutation $\pi$ by $\delta$ does not preserve this cyclic order. Thus, for a permutation $\pi$, the metric $\textup{dist}$ induces the Lee metric \cite[p.52]{Dist} on $\textup{Orb}(\pi)$:
$$
\textup{dist}(\pi r^i, \pi r^j) = \min \left( |i-j|,\ |n-(i-j)| \right).
$$

Let $\pi = [\pi_1 \dots \pi_{i-1}\pi_i\pi_{i+1}\dots\pi_{j-1}\pi_{j}\pi_{j+1}\dots\pi_n]
$.
Then, any permutation in $\textup{Orb}(\xi)$, where
$$\xi = [\pi_1 \dots \pi_{i-1}\pi_{j}\pi_{j-1} \dots \pi_{i+1}\pi_{i}\pi_{j+1}   \dots\pi_n],
$$
which reverses the order of elements within the subsequence $\pi_i,\pi_{i+1},\dots,\pi_j$, is called a \textit{$(\pi_i,\pi_j)$-reversal of length $j-i+1$ of the permutation $\pi$}.
Sometimes we simply use the terms \textit{reversal} or \textit{reversal of length $j-i+1$}.
The operation of moving from a permutation to its $(\pi_i,\pi_j)$-reversal is called \textit{reversing} the subsequence $\pi_i,\pi_{i+1},\dots,\pi_j$, or, more concisely, the subsequence $(\pi_i,\pi_j)$.
Thus, reversing the sequence $(\pi_1,\pi_n)$ of a permutation $\pi$ is equivalent to multiplying by $sr^x, x \in \mathbb{Z}_n$.






\section{A Lower Bound for the Diameter of Cayley Graph of the Symmetric Group $S_n$ Generated by  $(12), (12 \dots n), (1n \dots 2)$}



\begin{lemma}\label{lemma}
Let
$$
\pi = [\pi_1 \dots \pi_{j-1}\pi_j\pi_{j+1}\dots\pi_n],
$$
$$\xi = [\pi_{j}\pi_{j-1} \dots \pi_{1}\pi_{j+1} \dots \pi_n],
$$
where $2 \leqslant j \leqslant \cnt, n \geqslant 3.$
Then
\begin{equation}
\label{lem_1}
\textup{dist}(\pi,\xi r^{\fjt-1}) = j(j-1)-1, \tag{$\textup{I}$}
\end{equation}
\begin{equation}
\label{lem_2}
\textup{dist}(\pi r^{j-2},\xi r^{\cjt-1}) = j(j-1)-1, \tag{$\textup{II}$}
\end{equation}
\begin{equation}
\label{lem_3}
\textup{dist}(\pi r^{\fjt-1},\xi) = j(j-1)-1, \tag{$\textup{III}$}
\end{equation}
\begin{equation}
\label{lem_4}
\textup{dist}(\pi r^{\cjt-1},\xi r^{j-2}) = j(j-1)-1. \tag{$\textup{IV}$}
\end{equation}
\end{lemma}

\begin{proof}
We prove \textbf{Formula \ref{lem_1}}.
A permutation $\xi$ is the $(\pi_1,\pi_j)$-reversal of length $j$ of the permutation $\pi$, in which the elements $\pi_k$ and $\pi_l$ are swapped, with $1 \leqslant k \leqslant \fjt$, $l = j + 1 - k$, $k < l$.

The linear order $<$ defined on the set of indices $\mathbb{Z}_n$ naturally induces a linear order $\prec$ on the elements of the permutation $\pi$, which, in turn, induces the cyclic order $\prec_{cycle}$. That is, one may say that all elements of the permutation $\pi$ are in the cyclic order~$\prec_{cycle}$.
An \textit{inversion} is defined as a pair of elements $\pi_p$ and $\pi_q$ such that $\pi_p \nprec_{\pi_{j+1}} \pi_q$, where $p,q \in \mathbb{Z}_n$ and $p < q$, and where $\prec_{\pi_{j+1}}$ is the linear order determined by the element $\pi_{j+1}$.
The number of inversions of the sequence $\pi_1 \dots \pi_{j-1} \pi_j$ is equal to $0$, whereas the number of inversions of the sequence $\pi_j \pi_{j-1} \dots \pi_1$ is equal to $\frac{j(j-1)}{2}$.

Consider the permutation $\pi r^{k-1}$, whose first position is occupied by the element $\pi_k$. To optimally swap the elements $\pi_k$ and $\pi_l$ we use the following decomposition:
\begin{equation}
\label{reversal_1}
\pi r^{k-1}\left(\delta r\right)^{l-k-1}\delta\left(r^{-1}\delta\right)^{l-k-1} \tag{$\mathcal{A}$}.
\end{equation}

Optimality is ensured by the fact that each permutation $\delta$ can increase the number of inversions by at most one, and does so in this decomposition. Since $\delta^2 = ()$, there must be at least one shift between them, which is indeed present in the decomposition. Clearly, a shift preserves the number of inversions. The number of group elements in decomposition~\ref{reversal_1} is equal to $4(l-k-1)+1$.

To obtain $\xi r^{\fjt-1}$ one needs to swap $\fjt$ pairs of elements of the permutation. The minimal number of shifts required to move from one pair of elements to the next is $\fjt\!-\!1$.
Again, we note that shifts preserve the number of inversions.
Therefore, the total number of permutations, required to swap $\fjt$ pairs using decomposition \ref{reversal_1} is:
\[
\textup{dist}(\pi,\xi r^{\fjt-1}) = \sum_{k=1}^{\fjt} \bigl(4(l - k - 1) + 1\bigr) + \fjt - 1,
\quad \text{where } l = j + 1 - k.
\]

Substituting $l$ and simplifying the expression, we obtain the required estimate.

The total number of permutations $\delta$ when using decomposition~\ref{reversal_1} is
\[
\sum_{k=1}^{\fjt} \bigl(2(l - k - 1) + 1\bigr) = \frac{j(j-1)}{2},
\quad \text{where } l = j + 1 - k,
\]
which coincides with the number of inversions of the sequence $\pi_{j}\pi_{j-1} \dots \pi_{1}$.


Now we prove \textbf{Formula \ref{lem_2}}, that is, $\textup{dist}(\pi r^{j-2},\xi r^{\cjt-1}) = j(j-1)-1$.
Consider the permutation $\pi r^{l-2}$, whose first position is occupied by the element $\pi_{l-1}$. We obtain the following decomposition to swap the elements $\pi_k$ and $\pi_l$:
\begin{equation}
\label{reversal_4}
\pi r^{l-2}\left(\delta r^{-1}\right)^{l-k-1}
\delta
\left(r\delta\right)^{l-k-1}. \tag{$\mathcal{B}$}
\end{equation}

Decomposition \ref{reversal_4} is optimal and consists of $4(l-k-1)+1$ permutations. Therefore, the total number of operations is the same as in the previous case.
\textbf{Formulas \ref{lem_3}} and \textbf{\ref{lem_4}} are proved similarly.
For Formula \ref{lem_3}, i.e., $\textup{dist}(\pi r^{\fjt-1},\xi) = j(j-1)-1$, one should use decomposition \ref{reversal_1} to swap the elements of the permutation.
For Formula \ref{lem_4}, i.e., $\textup{dist}(\pi r^{\cjt-1},\xi r^{j-2}) = j(j-1)-1$, one should use decomposition \ref{reversal_4}, to swap the elements of the permutation.
This completes the proof of the lemma.
\end{proof}



\begin{theorem}\label{th_1}
Let $s$ $=$ $[n\ n\!-\!1\ \dots\ 1]$, $r = [2\ 3\ \dots\ n\ 1]$, $s,r \in S_n$, $n \geqslant 4$.
The following bound is valid:
$$
\textup{dist}(sr^{n-i},()) \! = \! \cnt \left(\cnt\!-\!1\right)-1 + \fnt\left(\fnt\!-\!1\right)-1 +
\begin{cases}
\fnt + 1, & i=1,\\[6pt]
\fnt - i + 4, & 2 \le \!i\! \le \fnt\!+\!2,\\[6pt]
i - \cnt, & \cnt\!+\!2 \le \!i\! \le n.
\end{cases}
$$
\end{theorem}
\begin{proof}




Since $sr^{n-i}sr^{n-i} = ()$, it is necessary to find the decomposition of the permutation $sr^{n-i}$.

Let us consider the elements of the permutation $s$ as elements of $\mathbb{Z}_n$.
A linear order $>$ is defined on them such that for any elements $s_p$ and $s_q$, where $p,q \in \mathbb{Z}_n$ and $p < q$, the relation $s_p > s_q$ holds. The order $>$ induces a cyclic order $>_{cycle}$. All elements of the permutation $s$ are in the relation $>_{cycle}$. Similarly, on the elements of the permutation $()$, considered as elements of $\mathbb{Z}_n$, the orders $<$ and $<_{cycle}$ are defined. Thus, one may say that the elements of the permutation $s$ are in the relation $<_{cycle}$ in the permutation~$()$. Moreover, as stated earlier in Section~\ref{sec::def}, the elements of the permutation $s$ are in the relation $<_{cycle}$ only in permutations from $\textup{Orb}(())$.
Therefore, it is necessary to compute $\textup{dist}(s,\textup{Orb}(()))$.

We divide the sequence of elements $n, n-1, \dots, 1$ of the permutation $s$ into two subsequences as follows.
The first subsequence $\textup{F}_1$ of length $\fnt$:
$$
n+m, n+m-1, \dots, n, \dots, n+m-\fnt+1,
$$
where $0 \leqslant m \leqslant n-1$.
The second subsequence $\textup{F}_2$ of length $\cnt$:
$$
n+m-\fnt, n+m-\fnt-1, \dots, 1, n, n-1, \dots, n+m+1.
$$


A simultaneous reversal of the subsequences $\textup{F}_1$ and $\textup{F}_2$ of the permutation $s$ is some permutation in $\textup{Orb}(())$, as follows from the definition.
Hence, it is necessary to perform reversals of the subsequences $\textup{F}_1$ and $\textup{F}_2$.
The Lemma~\ref{lemma} allows us to obtain the distances between permutations and their reversals.
Thus, a double application of Lemma \ref{lemma} allows one to determine the distance between the permutation $sr^{-m}$ and the permutation $sr^{-m}sr^x$, where $x\in\mathbb{Z}_n$.

For the first application of Lemma \ref{lemma}, we assume $j=\fnt$ and first consider those values of $m$ for which Lemma \ref{lemma} determines the distances between a permutation $s$ and some $(n+m,n+m-\fnt+1)$-reversal of the permutation $s$.
These values will be: $0,\fnt-2,\ffntt-1,\cfntt-1$.
Each of these values corresponds to one of the formulas \ref{lem_1}-\ref{lem_4}.
When choosing one of these values of $m$ and the corresponding formula \ref{lem_1}-\ref{lem_4}, the permutation $s$, which is the original one, will be substituted into the distance function as the first argument.
That is, the distance between $s$ and some $(n+m,n+m-\fnt+1)$-reversal of the permutation $s$ will be calculated.
Thus, a double application of the Lemma \ref{lemma} will allow us to determine the distance between the permutation $s$ and the permutation $r^x$.



If in one of the formulas \ref{lem_1}-\ref{lem_4}, when reversing $\textup{F}_1$, rather than the permutation $s$ being substituted, instead if it is the permutation $sr^y$, $y\in\mathbb{Z}_n$, $y\neq n$, then additional shifts of the permutation of $s$ will be required.
That is, the distance between the permutation $sr^y$ and the permutation $sr^ysr^x$ will be calculated.
However, since $sr^ysr^x = r^xr^{-y}$, without loss of generality, we will consider only the values of $m$ that are equal to one of the following: $0,\fnt-2,\ffntt-1,\cfntt-1$, and choose the appropriate formula \ref{lem_1}-\ref{lem_4} so that additional shifts of the permutation of $s$ are not needed.



In order to proceed from the reversal of the subsequence $\textup{F}_1$ to the reversal of the subsequence $\textup{F}_2$, it is necessary that at least two elements of $\textup{F}_2$ occupy the first positions of the permutation obtained after the first reversal.
This necessitates at least two shifts, taking into account the Lee metric.
There are only two possible situations in which two shifts will be enough. Let us consider them separately.

First, let $m = \cfntt - 1$. Then by the Lemma \ref{lemma} for $j=\fnt$ from the Formula \ref{lem_4} we obtain:
$$
\textup{dist}(sr^{-\left(\cfntt - 1\right)}r^{\cfntt - 1},\xi' r^{\fnt-2}) =
\textup{dist}(s,\xi' r^{\fnt-2}) = \fnt\left(\fnt-1\right)-1,
$$
where $\xi'$ is the permutation obtained from $sr^{-\left(\cfntt - 1\right)}$ by reversing $\textup{F}_1$.

In order to use the Lemma \ref{lemma} to reverse the subsequence $\textup{F}_2$ it is necessary and sufficient to apply two shifts to the element $\xi' r^{\fnt-2}$.
Thus we get the element $\xi' r^{\fnt}$.

Let us apply the Lemma \ref{lemma} again for $j=\cnt$.
From the Formula \ref{lem_1} we have:
$$
\textup{dist}(\xi' r^{\fnt},\xi'' r^{\fcntt-1}) = \cnt\left(\cnt-1\right)-1,
$$
where $\xi''$ is a permutation obtained from $\xi' r^{\fnt}$ by reversing $\textup{F}_2$, that is, $\xi'' = s$.

Thus, applying the Lemma \ref{lemma} for $m = \cfntt - 1$ twice to the permutation $sr^{-m}$, the permutation is obtained:
$$
r^{\cfntt - 1 + \fcntt - 1}.
$$


Now substitute the permutation $sr^{n-i}$ and simplify the expression:
$$
sr^{n-i}sr^{\cfntt - 1 + \fcntt - 1}
= r^{\cfntt - 1 + \fcntt - 1 - n + i}.
$$

Next, we use the Lee metric to calculate the distance:
$$
\textup{dist}(r^{\cfntt - 1 + \fcntt - 1 - n + i},()) =
\textup{dist}(r^{\cfntt - 1 + \fcntt - 1 - n + i},r^0) =
$$
$$
\min \left(\ \left|\cfntt \!-\! 1 \!+\! \fcntt \!-\! 1 \!-\! n \!+\! i\right|,\ \left|n\!-\!\left(\cfntt \!-\! 1 \!+\! \fcntt \!-\! 1 \!-\! n \!+\! i\right)\right|\ \right).
$$

Now, let $m = \ffntt - 1$ and perform similar actions.
Then by the Lemma \ref{lemma} for $j=\fnt$ from the Formula \ref{lem_3} we arrive at:
$$
\textup{dist}(sr^{-\left(\ffntt - 1\right)}r^{\ffntt - 1},\xi') =
\textup{dist}(s,\xi') = \fnt\left(\fnt-1\right)-1,
$$
where $\xi'$ is the permutation obtained from $sr^{-\left(\ffntt - 1\right)}$ by reversing $\textup{F}_1$.

In order to use the Lemma \ref{lemma} to reverse the subsequence $\textup{F}_2$ it is necessary and sufficient to apply two shifts to the element $\xi'$.
Thus, we end up with the element ${\xi'}^{-2}$.

Let us apply the Lemma \ref{lemma} again for $j=\cnt$.
From the Formula \ref{lem_2} we come to:
$$
\textup{dist}(\xi' r^{\cnt-2},
\xi'' r^{\ccntt-1}) = \cnt\left(\cnt-1\right)-1,
$$
where $\xi''$ is the permutation obtained from $\xi' r^{\cnt-2}$ by reversing $\textup{F}_2$, that is, $\xi''$~$=$~$s$.

Thus, applying the Lemma \ref{lemma} for $m = \ffntt - 1$ twice to the permutation $sr^{-m}$, the permutation is obtained
$$
r^{\ffntt - 1 + \ccntt - 1}.
$$


Now substitute the $sr^{n-i}$ and simplify the expression:
$$
sr^{n-i}sr^{\ffntt - 1 + \ccntt - 1}
= r^{\ffntt - 1 + \ccntt - 1 - n + i}.
$$

Next, we use the Lee metric to calculate the distance:
$$
\textup{dist}(r^{\ffntt - 1 + \ccntt - 1 - n + i},()) =
\textup{dist}(r^{\ffntt - 1 + \ccntt - 1 - n + i},r^0) =
$$
$$
\min \left(\ \left|\ffntt \!-\! 1 \!+\! \ccntt \!-\! 1 \!-\! n \!+\! i\right|,\ \left|n\!-\!\left(\ffntt \!-\! 1 \!+\! \ccntt \!-\! 1 \!-\! n \!+\! i\right)\right|\ \right).
$$


Of the two cases considered, it is necessary to choose a decomposition with a minimum number of elements. Thus, we work out the total number of decomposition elements:
\[
\begin{aligned}
&\quad\quad\quad \cnt \left(\cnt-1\right)-1
 + \fnt\left(\fnt-1\right)-1
 + 2 \\
&+ \min\left(
\begin{aligned}
  &\left|\cfntt - 1 + \fcntt - 1 - n + i\right|,\\
  &\left|n-\left(\cfntt - 1 + \fcntt - 1 - n + i\right)\right|,\\
  &\left|\ffntt - 1 + \ccntt - 1 - n + i\right|,\\
  &\left|n-\left(\ccntt - 1 + \ffntt - 1 - n + i\right)\right|
\end{aligned}
\right).
\end{aligned}
\]

By simplifying the formula, we calculate the final result.
The theorem has been proved.

\end{proof}




\begin{theorem}
A lower bound for the diameter of Cayley graph of the symmetric group $S_n$ generated by  $(12), (12 \dots n), (1n \dots 2)$ is $\frac{n(n-1)}{2}$ for $n \geqslant 4$.
\end{theorem}

\begin{proof}
The lower bound for the graph diameter follows from Theorem~\ref{th_1} for $i=2$.
This completes the proof.
\end{proof}





\textbf{Acknowledgments.} The author expresses gratitude to Alexander Chervov (Institut Curie, Paris) for posing the problem and for the invitation to participate in the CayleyPy project.





\begin{thebibliography}{99}

\bibitem{Alon}
R.\,M.~Adin, N.~Alon, Y.~Roichman,
``Circular sorting,''
\href{https://arxiv.org/abs/2502.14398}{arXiv:2502.14398}, 2025.

\bibitem{SawadaWilliams19}
J.~Sawada, A.~Williams,
``Solving the Sigma-Tau Problem,''
\textit{ACM Transactions on Algorithms}, vol.~16, no.~1, pp.~1--17, 2019.

\bibitem{Zubov}
A.~Zubov,
``On the diameter of the group $S_n$ relative to a system of generation consisting of a full cycle and transposition,''
\textit{Studies on Discrete Mathematics}, Russian Academy of Sciences, Academy of Cryptography of Russian Federation, vol.~2, pp.~112--150, 1998.

\bibitem{ChervovRL}
A.\,A.~Chervov, A.\,V.~Soibelman, S.\,A.~Lytkin, I.\,A.~Kiselev, S.\,V.~Fironov, A.\,A.~Lukyanenko, A.\,A.~Dolgorukova, A.\,A.~Ogurtsov, F.\,V.~Petrov, S.\,A.~Krymskii, M.\,S.~Evseev, L.\,A.~Grunwald, D.\,A.~Gorodkov, G.\,A.~Antiufeev, G.\,A.~Verbii, V.\,A.~Zamkovoy, L.\,A.~Cheldieva, I.\,A.~Koltsov, A.\,A.~Sychev, M.\,S.~Obozov, A.\,A.~Eliseev, S.\,I.~Nikolenko, N.\,A.~Narynbaev, R.\,R.~Turtayev, N.\,A.~Rokotyan, S.\,V.~Kovalev, A.\,A.~Rozanov, V.\,A.~Nelin, S.\,A.~Ermilov, L.\,A.~Shishina, D.\,A.~Mamayeva, A.\,A.~Korolkova, K.\,A.~Khoruzhii, A.\,A.~Romanov,
``CayleyPy RL: Pathfinding and Reinforcement Learning on Cayley Graphs,''
\href{https://arxiv.org/abs/2502.18663}{arXiv:2502.18663}, 2025.

\bibitem{Konstantinova}
E.\,V.~Konstantinova,
\textit{Lectures on Algebraic Graph Theory: a Textbook},
Novosibirsk: Publishing and Printing Center (IPC), Novosibirsk State University, 2023.

\bibitem{Hunt1935}
V.~Huntington,
``Inter-Relations Among the Four Principal Types of Order,''
\textit{Transactions of the American Mathematical Society}, vol.~38, no.~1, pp.~1--9, 1935.

\bibitem{Dist}
E.~Deza, M.~Deza,
\textit{Dictionary of Distances} (3rd ed.),
Heidelberg, New York, Dordrecht, London: Springer, 2014.

\end{thebibliography}

\end{document}
