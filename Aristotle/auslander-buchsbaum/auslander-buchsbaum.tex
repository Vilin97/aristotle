\begin{lemma}
\label{lemma-regular-local-UFD}
A regular local ring is a UFD.
\end{lemma}

\begin{proof}
Recall that a regular local ring is a domain, see
Algebra, Lemma \ref{algebra-lemma-regular-domain}.
We will prove the unique factorization property
by induction on the dimension of the regular local ring $R$.
If $\dim(R) = 0$, then $R$ is a field and in particular a UFD.
Assume $\dim(R) > 0$. Let $x \in \mathfrak m$, $x \not \in \mathfrak m^2$.
Then $R/(x)$ is regular by Algebra, Lemma \ref{algebra-lemma-regular-ring-CM},
hence a domain by
Algebra, Lemma \ref{algebra-lemma-regular-domain},
hence $x$ is a prime element.
Let $\mathfrak p \subset R$ be a height $1$ prime. We have
to show that $\mathfrak p$ is principal, see
Algebra, Lemma \ref{algebra-lemma-characterize-UFD-height-1}.
We may assume $x \not \in \mathfrak p$, since if $x \in \mathfrak p$,
then $\mathfrak p = (x)$ and we are done.
For every nonmaximal prime $\mathfrak q \subset R$
the local ring $R_\mathfrak q$ is a regular local ring, see
Algebra, Lemma \ref{algebra-lemma-localization-of-regular-local-is-regular}.
By induction we see that $\mathfrak pR_\mathfrak q$ is principal.
In particular, the $R_x$-module $\mathfrak p_x = \mathfrak pR_x \subset R_x$
is a finitely presented $R_x$-module whose localization at
any prime is free of rank $1$. 
By Algebra, Lemma \ref{algebra-lemma-finite-projective}
we see that $\mathfrak p_x$ is an invertible $R_x$-module.
By Lemma \ref{lemma-regular-local-Pic-zero} we see that
$\mathfrak p_x = (y)$ for some $y \in R_x$.
We can write $y = x^e f$ for some $f \in \mathfrak p$ and $e \in \mathbf{Z}$.
Factor $f = a_1 \ldots a_r$ into irreducible elements of $R$
(Algebra, Lemma \ref{algebra-lemma-factorization-exists}).
Since $\mathfrak p$ is prime, we see that $a_i \in \mathfrak p$
for some $i$. Since $\mathfrak p_x = (y)$ is prime and
$a_i | y$ in $R_x$, it follows that $\mathfrak p_x$ is generated by
$a_i$ in $R_x$, i.e., the image of $a_i$ in $R_x$ is prime.
As $x$ is a prime element, we find that $a_i$ is prime in $R$ by
Algebra, Lemma \ref{algebra-lemma-invert-prime-elements}.
Since $(a_i) \subset \mathfrak p$ and $\mathfrak p$ has height
$1$ we conclude that $(a_i) = \mathfrak p$ as desired.
\end{proof}