\documentclass[11pt]{article}

\usepackage{amsmath,amssymb,amsthm,mathtools}

% ---------- basic notation ----------
\newcommand{\Sn}{S_n}
\newcommand{\id}{()}
\newcommand{\dist}{\mathrm{dist}}

% generators
\newcommand{\del}{\delta}

\theoremstyle{plain}
\newtheorem{theorem}{Theorem}
\newtheorem{lemma}{Lemma}
\theoremstyle{definition}
\newtheorem{definition}{Definition}

\begin{document}

\begin{center}
{\Large Lean-friendly proof document: detailed Lee distance on $\langle r\rangle$}
\end{center}

\section{Setup and goal}

Fix an integer $n\ge 3$. Write permutations $\pi\in S_n$ in one-line notation
\[
\pi=[\pi_1\ \pi_2\ \cdots\ \pi_n],
\]
meaning $\pi(i)=\pi_i$ for $i=1,\dots,n$.

Define
\[
\del:=(12)\in S_n,\qquad
r := [2\ 3\ \cdots\ n\ 1]\in S_n,\qquad
r^{-1} := [n\ 1\ 2\ \cdots\ n-1].
\]

We work in the (unweighted) Cayley graph of $S_n$ with generating set $\{\del,r,r^{-1}\}$, and
$\dist(\pi,\xi)$ denotes the shortest word length in $\{\del,r,r^{-1}\}$ sending $\pi$ to $\xi$
by right-multiplication.

\subsection*{Right-multiplication formulas (used repeatedly)}
For $\pi=[\pi_1\cdots\pi_n]$:
\begin{align}
\pi \del &= [\pi_2\ \pi_1\ \pi_3\ \cdots\ \pi_n], \label{eq:delta-action}\\
\pi r &= [\pi_2\ \pi_3\ \cdots\ \pi_n\ \pi_1], \label{eq:r-action}\\
\pi r^{-1} &= [\pi_n\ \pi_1\ \pi_2\ \cdots\ \pi_{n-1}]. \label{eq:rinv-action}
\end{align}

\section{An index function and a cycle metric}

\begin{definition}[Position of the symbol $1$]
For $\pi=[\pi_1\cdots\pi_n]\in S_n$, define $\operatorname{pos}_1(\pi)\in\{1,\dots,n\}$
to be the unique index $i$ such that $\pi_i=1$.
\end{definition}

\begin{lemma}[Well-definedness of $\operatorname{pos}_1$]\label{lem:pos1-well-defined}
For every $\pi\in S_n$, there exists a unique $i\in\{1,\dots,n\}$ with $\pi_i=1$.
\end{lemma}

\begin{proof}
By definition of one-line notation, the entries $\pi_1,\dots,\pi_n$ are a permutation of
$\{1,\dots,n\}$, hence $1$ occurs at least once. Since all entries are distinct, it occurs
at most once. Therefore it occurs exactly once, giving existence and uniqueness.
\end{proof}

\begin{definition}[Cycle distance on indices]\label{def:cyc-dist}
For $i,j\in\{1,\dots,n\}$ define
\[
d_n(i,j):=\min\bigl(|i-j|,\ n-|i-j|\bigr).
\]
\end{definition}

\begin{lemma}[Elementary properties of $d_n$]\label{lem:dn-basic}
For all $i,j,k\in\{1,\dots,n\}$:
\begin{enumerate}
\item $d_n(i,j)\ge 0$ and $d_n(i,j)=0$ iff $i=j$.
\item $d_n(i,j)=d_n(j,i)$.
\item (Triangle inequality) $d_n(i,k)\le d_n(i,j)+d_n(j,k)$.
\item If $d_n(i,j)\le 1$ and $d_n(j,k)\le 1$, then $d_n(i,k)\le 2$ (a special case of (3)).
\end{enumerate}
\end{lemma}

\begin{proof}
These are standard facts for the shortest-path distance on the cycle graph $C_n$:
vertices are $\{1,\dots,n\}$ and edges connect $t$ to $t+1$ (and $n$ to $1$).
The quantity $d_n(i,j)$ is exactly the length of the shorter of the two cyclic arcs
from $i$ to $j$, hence satisfies (1)--(3). Item (4) follows immediately from (3).
\end{proof}

\section{How $\operatorname{pos}_1$ changes under generators}

\begin{lemma}[Effect of right-multiplication by $r$ and $r^{-1}$ on $\operatorname{pos}_1$]\label{lem:pos1-r}
Let $\pi=[\pi_1\cdots\pi_n]\in S_n$ and set $i:=\operatorname{pos}_1(\pi)$ (so $\pi_i=1$).
\begin{enumerate}
\item If $i\ge 2$, then $\operatorname{pos}_1(\pi r)=i-1$. If $i=1$, then $\operatorname{pos}_1(\pi r)=n$.
\item If $i\le n-1$, then $\operatorname{pos}_1(\pi r^{-1})=i+1$. If $i=n$, then $\operatorname{pos}_1(\pi r^{-1})=1$.
\end{enumerate}
In particular, in both cases we have
\[
d_n\bigl(\operatorname{pos}_1(\pi),\operatorname{pos}_1(\pi r)\bigr)=1,\qquad
d_n\bigl(\operatorname{pos}_1(\pi),\operatorname{pos}_1(\pi r^{-1})\bigr)=1.
\]
\end{lemma}

\begin{proof}
(1) By \eqref{eq:r-action}, $\pi r=[\pi_2\ \pi_3\ \cdots\ \pi_n\ \pi_1]$.
So the entry $1$ moves one position to the left (cyclically):
if $\pi_i=1$ with $i\ge 2$, then in $\pi r$ the value $1$ appears in position $i-1$.
If $i=1$ (meaning $\pi_1=1$), then in $\pi r$ the last entry is $\pi_1=1$, so its position is $n$.

(2) By \eqref{eq:rinv-action}, $\pi r^{-1}=[\pi_n\ \pi_1\ \cdots\ \pi_{n-1}]$.
So the entry $1$ moves one position to the right (cyclically):
if $\pi_i=1$ with $i\le n-1$, then in $\pi r^{-1}$ the value $1$ appears in position $i+1$.
If $i=n$, then in $\pi r^{-1}$ the first entry is $\pi_n=1$, so its position is $1$.

The $d_n$ equalities follow because in either case the index changes to the cyclic neighbor.
\end{proof}

\begin{lemma}[Effect of right-multiplication by $\del$ on $\operatorname{pos}_1$]\label{lem:pos1-delta}
Let $\pi\in S_n$ and $i:=\operatorname{pos}_1(\pi)$.
Then $\operatorname{pos}_1(\pi\del)=i$ if $i\notin\{1,2\}$, while
$\operatorname{pos}_1(\pi\del)=3-i$ if $i\in\{1,2\}$ (i.e.\ $1\leftrightarrow 2$ is swapped).
In particular,
\[
d_n\bigl(\operatorname{pos}_1(\pi),\operatorname{pos}_1(\pi\del)\bigr)\le 1.
\]
\end{lemma}

\begin{proof}
By \eqref{eq:delta-action}, $\pi\del=[\pi_2\ \pi_1\ \pi_3\ \cdots\ \pi_n]$.
So positions $3,\dots,n$ are unchanged, while the first two entries are swapped.
If $i\ge 3$, then $\pi_i=1$ remains in position $i$, hence $\operatorname{pos}_1(\pi\del)=i$.
If $i=1$, then $\pi_1=1$ becomes the second entry of $\pi\del$, so the new position is $2$.
If $i=2$, then $\pi_2=1$ becomes the first entry of $\pi\del$, so the new position is $1$.
Thus when $i\in\{1,2\}$, the new position is exactly $3-i$.
The $d_n$ bound is immediate: either the index is unchanged (distance $0$) or it swaps $1\leftrightarrow 2$ (distance $1$).
\end{proof}

\section{A general lower bound on word length via $\operatorname{pos}_1$}

\begin{lemma}[Any generator step changes $\operatorname{pos}_1$ by at most $1$]\label{lem:one-step}
For any $\pi\in S_n$ and any generator $g\in\{\del,r,r^{-1}\}$,
\[
d_n\bigl(\operatorname{pos}_1(\pi),\operatorname{pos}_1(\pi g)\bigr)\le 1.
\]
\end{lemma}

\begin{proof}
For $g=r$ or $g=r^{-1}$ this is Lemma~\ref{lem:pos1-r}. For $g=\del$ this is Lemma~\ref{lem:pos1-delta}.
\end{proof}

\begin{lemma}[Path-length lower bound]\label{lem:path-lower-bound}
Let $\pi,\xi\in S_n$. If there exists a word $g_1\cdots g_L$ with each $g_t\in\{\del,r,r^{-1}\}$ and
$\pi g_1\cdots g_L=\xi$, then
\[
d_n\bigl(\operatorname{pos}_1(\pi),\operatorname{pos}_1(\xi)\bigr)\le L.
\]
Consequently,
\[
\dist(\pi,\xi)\ \ge\ d_n\bigl(\operatorname{pos}_1(\pi),\operatorname{pos}_1(\xi)\bigr).
\]
\end{lemma}

\begin{proof}
Define intermediate vertices $\pi^{(0)}:=\pi$ and $\pi^{(t)}:=\pi g_1\cdots g_t$ for $t=1,\dots,L$,
so that $\pi^{(L)}=\xi$.
By Lemma~\ref{lem:one-step}, for each $t$ we have
\[
d_n\bigl(\operatorname{pos}_1(\pi^{(t-1)}),\operatorname{pos}_1(\pi^{(t)})\bigr)\le 1.
\]
Applying the triangle inequality (Lemma~\ref{lem:dn-basic}(3)) repeatedly gives
\[
d_n\bigl(\operatorname{pos}_1(\pi),\operatorname{pos}_1(\xi)\bigr)
\le \sum_{t=1}^L d_n\bigl(\operatorname{pos}_1(\pi^{(t-1)}),\operatorname{pos}_1(\pi^{(t)})\bigr)
\le \sum_{t=1}^L 1
= L.
\]
Since $\dist(\pi,\xi)$ is the minimum such $L$, the stated lower bound follows.
\end{proof}

\section{Computing $\operatorname{pos}_1(r^t)$ explicitly}

\begin{lemma}[Explicit one-line form of $r^t$ and $\operatorname{pos}_1(r^t)$]\label{lem:pos1-rt}
For each integer $t\ge 0$, the permutation $r^t$ in one-line notation is the left shift by $t$:
\[
r^t=[t+1\ \ t+2\ \ \cdots\ \ n\ \ 1\ \ 2\ \ \cdots\ \ t],
\]
where entries are interpreted in $\{1,\dots,n\}$ in the obvious way.
In particular,
\[
\operatorname{pos}_1(r^t)=
\begin{cases}
1,& t\equiv 0\ (\mathrm{mod}\ n),\\
n-(t\bmod n)+1,& t\not\equiv 0\ (\mathrm{mod}\ n),
\end{cases}
\]
equivalently: $\operatorname{pos}_1(r^t)$ is the representative in $\{1,\dots,n\}$ of $1-t$ modulo $n$.
\end{lemma}

\begin{proof}
We prove the displayed one-line form by induction on $t\ge 0$.

Base case $t=0$: $r^0=\id=[1\ 2\ \cdots\ n]$, which matches the formula.

Inductive step: assume
$r^t=[t+1\ \ t+2\ \cdots\ n\ 1\ 2\ \cdots\ t]$.
Then using \eqref{eq:r-action} with $\pi=r^t$, we have
\[
r^{t+1}=r^t r = [(r^t)_2\ (r^t)_3\ \cdots\ (r^t)_n\ (r^t)_1],
\]
i.e.\ we left-shift the one-line notation by one position, which produces exactly
\[
r^{t+1}=[t+2\ \ t+3\ \cdots\ n\ 1\ 2\ \cdots\ t\ \ t+1],
\]
matching the claimed formula for $t+1$.

For $\operatorname{pos}_1(r^t)$: in the displayed one-line notation, the entry $1$ occurs after the block
$t+1,\dots,n$, which has length $n-t$ when $0\le t\le n$.
Thus for $1\le t\le n-1$, the position is $n-t+1$.
For general $t\ge 0$, reduce $t$ modulo $n$ because $r^n=\id$.
Rephrasing, $\operatorname{pos}_1(r^t)$ is the unique representative of $1-t$ modulo $n$ lying in $\{1,\dots,n\}$.
\end{proof}

\section{Lee distance on $\langle r\rangle$}

\begin{definition}[Reduced difference and Lee distance on exponents]\label{def:lee-exp}
Fix integers $a,b$.
Let $d\in\{0,1,\dots,n-1\}$ be the remainder of $a-b$ modulo $n$ (i.e.\ $d\equiv a-b\ (\mathrm{mod}\ n)$).
Define
\[
\mathrm{Lee}_n(a,b):=\min(d,\ n-d).
\]
\end{definition}

\begin{lemma}[Easy upper bound using only $r^{\pm1}$]\label{lem:lee-upper}
For all integers $a,b$,
\[
\dist(r^a,r^b)\le \mathrm{Lee}_n(a,b).
\]
\end{lemma}

\begin{proof}
Let $d\in\{0,\dots,n-1\}$ be as in Definition~\ref{def:lee-exp}.
Then $r^a(r^{-1})^d=r^{a-d}=r^b$ in $S_n$ because $a-d\equiv b\ (\mathrm{mod}\ n)$.
This gives a word of length $d$ from $r^a$ to $r^b$.
Similarly, $r^a r^{n-d}=r^{a+n-d}=r^b$ gives a word of length $n-d$.
Therefore $\dist(r^a,r^b)\le \min(d,n-d)=\mathrm{Lee}_n(a,b)$.
\end{proof}

\begin{lemma}[Matching lower bound]\label{lem:lee-lower}
For all integers $a,b$,
\[
\dist(r^a,r^b)\ge \mathrm{Lee}_n(a,b).
\]
\end{lemma}

\begin{proof}
By Lemma~\ref{lem:path-lower-bound},
\[
\dist(r^a,r^b)\ \ge\ d_n\bigl(\operatorname{pos}_1(r^a),\operatorname{pos}_1(r^b)\bigr).
\]
By Lemma~\ref{lem:pos1-rt}, $\operatorname{pos}_1(r^t)$ is congruent to $1-t$ modulo $n$.
Hence the cyclic distance between $\operatorname{pos}_1(r^a)$ and $\operatorname{pos}_1(r^b)$
depends only on $(1-a)-(1-b)=b-a$ modulo $n$, i.e.\ on $a-b$ modulo $n$.
Concretely, let $d\in\{0,\dots,n-1\}$ be the remainder of $a-b$ modulo $n$.
Then moving from $\operatorname{pos}_1(r^a)$ to $\operatorname{pos}_1(r^b)$ along the cycle requires
either $d$ steps in one direction or $n-d$ steps in the other, so
\[
d_n\bigl(\operatorname{pos}_1(r^a),\operatorname{pos}_1(r^b)\bigr)=\min(d,n-d)=\mathrm{Lee}_n(a,b).
\]
Therefore $\dist(r^a,r^b)\ge \mathrm{Lee}_n(a,b)$.
\end{proof}

\begin{lemma}[Lee distance on $\langle r\rangle$ (correct integer statement)]\label{lem:lee}
For all integers $a,b$,
\[
\dist(r^a,r^b)=\mathrm{Lee}_n(a,b),
\]
where $\mathrm{Lee}_n(a,b)$ is defined using the reduced remainder $d\in\{0,\dots,n-1\}$ of $a-b$ modulo $n$:
$\mathrm{Lee}_n(a,b)=\min(d,n-d)$.

In particular, for any integer $t$,
\[
\dist(r^t,\id)=\mathrm{Lee}_n(t,0).
\]
Moreover, if $a,b$ are such that $0\le |a-b|\le n$, then $\mathrm{Lee}_n(a,b)=\min(|a-b|,\ n-|a-b|)$,
so the familiar informal formula holds in that range.
\end{lemma}

\begin{proof}
Combine Lemma~\ref{lem:lee-upper} and Lemma~\ref{lem:lee-lower}.
\end{proof}

\end{document}
