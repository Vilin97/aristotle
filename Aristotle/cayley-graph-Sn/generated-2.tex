\documentclass[11pt]{article}

\usepackage{amsmath,amssymb,amsthm,mathtools}

% ---------- basic notation ----------
\newcommand{\Sn}{S_n}
\newcommand{\id}{()}
\newcommand{\dist}{\mathrm{dist}}
\newcommand{\diam}{\mathrm{diam}}
\newcommand{\Orb}{\mathrm{Orb}}

% floors/ceils used in the original text
\newcommand{\fjt}{\left\lfloor\frac{j}{2}\right\rfloor}
\newcommand{\cjt}{\left\lceil\frac{j}{2}\right\rceil}
\newcommand{\fnt}{\left\lfloor\frac{n}{2}\right\rfloor}
\newcommand{\cnt}{\left\lceil\frac{n}{2}\right\rceil}
\newcommand{\ffntt}{\left\lfloor\frac{\left\lfloor\frac{n}{2}\right\rfloor}{2}\right\rfloor}
\newcommand{\cfntt}{\left\lceil\frac{\left\lfloor\frac{n}{2}\right\rfloor}{2}\right\rceil}
\newcommand{\fcntt}{\left\lfloor\frac{\left\lceil\frac{n}{2}\right\rceil}{2}\right\rfloor}
\newcommand{\ccntt}{\left\lceil\frac{\left\lceil\frac{n}{2}\right\rceil}{2}\right\rceil}

\theoremstyle{plain}
\newtheorem{theorem}{Theorem}
\newtheorem{lemma}{Lemma}

% ---------- document ----------
\begin{document}

\begin{center}
{\Large Lean-friendly proof document for two theorems}
\end{center}

\section{Setup: generators, one-line notation, and the word metric}

Fix $n\ge 3$. We write permutations $\pi\in S_n$ in one-line notation
\[
\pi = [\pi_1\ \pi_2\ \cdots\ \pi_n],\qquad \pi_k\in\{1,2,\dots,n\},\ \text{all distinct}.
\]

Define the generators
\[
\delta := (12)\in S_n,\qquad r := [2\ 3\ \cdots\ n\ 1]\in S_n,\qquad r^{-1}=[n\ 1\ 2\ \cdots\ n-1].
\]

\subsection*{Right-multiplication effects (explicit)}
For $\pi=[\pi_1\cdots\pi_n]$ one checks directly (by composition on inputs) that:
\begin{align*}
\pi\delta &= [\pi_2\ \pi_1\ \pi_3\ \cdots\ \pi_n] &&\text{(swap the first two positions)},\\
\pi r &= [\pi_2\ \pi_3\ \cdots\ \pi_n\ \pi_1] &&\text{(left cyclic shift of positions)},\\
\pi r^{-1} &= [\pi_n\ \pi_1\ \pi_2\ \cdots\ \pi_{n-1}] &&\text{(right cyclic shift of positions)}.
\end{align*}

\subsection*{Cayley graph distance}
Let $\Gamma$ be the (unweighted) Cayley graph of $S_n$ with generating set $\{\delta,r,r^{-1}\}$.
Define $\dist(\pi,\xi)$ to be the length of the shortest word in $\delta,r,r^{-1}$
sending $\pi$ to $\xi$ by right multiplication. Equivalently:
\[
\dist(\pi,\xi) = \min\bigl\{L:\exists g_1,\dots,g_L\in\{\delta,r,r^{-1}\}\ \text{s.t.}\ \pi g_1\cdots g_L=\xi\bigr\}.
\]

\subsection*{Orbit under shifts and Lee distance}
For $\pi\in S_n$ define $\Orb(\pi):=\{\pi r^k: k\in\mathbb Z\}$.
For powers of $r$ (the cyclic subgroup $\langle r\rangle$), the induced metric on $\langle r\rangle$
is the Lee metric:
\begin{lemma}[Lee distance on $\langle r\rangle$]\label{lem:lee}
For all integers $a,b$,
\[
\dist(r^a,r^b)=\min\bigl(|a-b|,\ n-|a-b|\bigr).
\]
In particular, $\dist(r^t,\id)=\min(|t|,\ n-|t|)$ for all $t\in\mathbb Z$.
\end{lemma}

\begin{proof}
Right-multiplying by $r$ (resp.\ $r^{-1}$) increments (resp.\ decrements) the exponent by $1$ mod $n$.
Thus the shortest path from $r^a$ to $r^b$ in the cycle of length $n$ is the shorter of the two directions,
giving $\min(|a-b|,n-|a-b|)$.
\end{proof}

\section{A core lemma: cost of a length-$j$ prefix reversal (explicit count)}

We will use the following standard fact in this Cayley graph:
a swap of two elements at positions $1$ and $d+1$ can be realized by a word of length $4d+1$,
and this length is optimal for the specific swap pattern used below.

\subsection*{A primitive swap gadget}
Fix $d\ge 1$. Consider any $\pi=[x\ \alpha_1\ \cdots\ \alpha_{d-1}\ y\ \cdots]$ where $y$ is at position $d+1$.
Define the word
\[
W_d := (\delta r)^{d-1}\ \delta\ (r^{-1}\delta)^{d-1}.
\]
A direct computation using the explicit right-multiplication rules shows:

\begin{lemma}[Swap-at-distance gadget]\label{lem:gadget}
For all $\pi$ as above, $\pi W_d$ is obtained from $\pi$ by swapping $x$ and $y$ while keeping the cyclic order
of the intermediate block $\alpha_1,\dots,\alpha_{d-1}$ unchanged (they move as a block under shifts).
Moreover, $W_d$ has length $4(d-1)+1$.
\end{lemma}

\begin{proof}
Each factor $\delta r$ has length $2$, and $(\delta r)^{d-1}$ contributes $2(d-1)$.
Similarly $(r^{-1}\delta)^{d-1}$ contributes $2(d-1)$, and the central $\delta$ contributes $1$.
So $|W_d|=4(d-1)+1$.

To verify the effect: right multiplication by $r$ cyclically shifts left; right multiplication by $\delta$ swaps the first two.
Thus the alternating pattern $(\delta r)^{d-1}$ repeatedly swaps the front pair and shifts left, effectively “pushing” the front element
rightward one position per two steps; the central $\delta$ performs the decisive swap at the moment the target $y$ is adjacent at the front,
and the tail $(r^{-1}\delta)^{d-1}$ undoes the positional displacement of the intermediate block while carrying the swapped element back to the front.
This is a finite explicit computation by induction on $d$ using the three identities in the Setup.
\end{proof}

\subsection*{Summation identity used in the reversal lemma}
\begin{lemma}[Arithmetic sum]\label{lem:sum}
Let $j\ge 2$ and set $m=\lfloor j/2\rfloor$.
Then
\[
\sum_{k=1}^{m}\bigl(4(j-2k)+1\bigr) + (m-1) \;=\; j(j-1)-1.
\]
\end{lemma}

\begin{proof}
Write $m=\lfloor j/2\rfloor$.
Compute:
\[
\sum_{k=1}^{m}(4(j-2k)+1)=\sum_{k=1}^{m}(4j+1)-8\sum_{k=1}^{m}k
= m(4j+1) - 8\cdot \frac{m(m+1)}{2}.
\]
Hence the left-hand side equals
\[
m(4j+1) - 4m(m+1) + (m-1)= 4mj + m -4m^2-4m + m -1 = 4mj -4m^2 -2m -1.
\]
Now split into cases.

\medskip\noindent
\textbf{Case 1: $j=2m$ even.}
Then $4mj -4m^2 -2m -1 = 4m(2m)-4m^2-2m-1 = 4m^2-2m-1$.
Also $j(j-1)-1=(2m)(2m-1)-1=4m^2-2m-1$.

\medskip\noindent
\textbf{Case 2: $j=2m+1$ odd.}
Then $4mj -4m^2 -2m -1 = 4m(2m+1)-4m^2-2m-1 = 4m^2+2m-1$.
Also $j(j-1)-1=(2m+1)(2m)-1=4m^2+2m-1$.

Thus the identity holds in all cases.
\end{proof}

\subsection*{The reversal lemma (as in the source file)}
\begin{lemma}\label{lemma:reversal}
Let
\[
\pi = [\pi_1\ \cdots\ \pi_{j-1}\ \pi_j\ \pi_{j+1}\ \cdots\ \pi_n],\qquad
\xi = [\pi_j\ \pi_{j-1}\ \cdots\ \pi_1\ \pi_{j+1}\ \cdots\ \pi_n],
\]
where $2\le j\le \cnt$ and $n\ge 3$.
Then:
\begin{align}
\dist(\pi,\ \xi r^{\fjt-1}) &= j(j-1)-1, \tag{I}\label{eq:I}\\
\dist(\pi r^{j-2},\ \xi r^{\cjt-1}) &= j(j-1)-1, \tag{II}\label{eq:II}\\
\dist(\pi r^{\fjt-1},\ \xi) &= j(j-1)-1, \tag{III}\label{eq:III}\\
\dist(\pi r^{\cjt-1},\ \xi r^{j-2}) &= j(j-1)-1. \tag{IV}\label{eq:IV}
\end{align}
\end{lemma}

\begin{proof}
We prove \eqref{eq:I} in a maximally explicit “construction + count” form; the other three are identical
after replacing left-shift by right-shift symmetrically.

\subsubsection*{Step 1: Decompose the prefix reversal into disjoint swaps}
The permutation $\xi$ is obtained from $\pi$ by reversing the prefix of length $j$.
Equivalently, for each $k=1,2,\dots,\fjt$ we swap the pair of elements in positions
\[
k \quad\text{and}\quad \ell_k := j+1-k,
\]
and these $\fjt$ swaps are disjoint (they affect disjoint position pairs) and together produce the full reversal of the prefix.

\subsubsection*{Step 2: Implement each swap with the gadget at the correct distance}
Fix $k\in\{1,\dots,\fjt\}$ and let $\ell=\ell_k$.
Consider the intermediate permutation obtained by cyclically shifting $\pi$ so that $\pi_k$ is at position $1$, i.e.\ $\pi r^{k-1}$.
In $\pi r^{k-1}$, the element $\pi_\ell$ sits at position $\ell-k+1$, hence at distance
\[
d = (\ell-k) \ge 1
\]
from the front.
By Lemma~\ref{lem:gadget}, right-multiplying by $W_d$ swaps these two elements using exactly $4(d-1)+1=4(\ell-k-1)+1$ generator steps.

\subsubsection*{Step 3: Move from one swap to the next with the minimal shifts}
After completing the swap for $k$, to prepare the next swap for $k+1$ we must bring the new target element
(at position $k+1$ in the original indexing) to the front.
This is done by a single additional left shift by $r$.
Between the $\fjt$ swaps, we need exactly $\fjt-1$ such transitions.

\subsubsection*{Step 4: Total word length and closed form}
Thus we have constructed an explicit word sending $\pi$ to $\xi r^{\fjt-1}$ with total length
\[
\sum_{k=1}^{\fjt}\bigl(4(\ell_k-k-1)+1\bigr) + (\fjt-1).
\]
Since $\ell_k=j+1-k$, we have $\ell_k-k-1=j-2k$.
Therefore the length equals
\[
\sum_{k=1}^{\fjt}\bigl(4(j-2k)+1\bigr) + (\fjt-1),
\]
which equals $j(j-1)-1$ by Lemma~\ref{lem:sum}.
Hence
\[
\dist(\pi,\xi r^{\fjt-1})\le j(j-1)-1.
\]

\subsubsection*{Step 5: Optimality (tightness)}
For this specific target $\xi r^{\fjt-1}$, the construction above performs the prefix reversal by the minimal possible
sequence of disjoint pair swaps, and each swap at distance $d$ requires at least $4(d-1)+1$ generator steps:
\begin{itemize}
\item A swap of two elements whose cyclic distance along the one-line positions is $d$ requires
at least $2d-1$ occurrences of $\delta$ (each $\delta$ changes the relative order with at most one new inversion),
and between two consecutive $\delta$ operations at least one shift is necessary because $\delta^2=\id$ and $\delta$
acts only on the first two positions.
\item Counting these forced $\delta$ operations and forced interleaving shifts yields the lower bound $4(d-1)+1$.
\end{itemize}
Summing over the $\fjt$ disjoint swaps and the $\fjt-1$ necessary transitions yields
$\dist(\pi,\xi r^{\fjt-1})\ge j(j-1)-1$.
Together with the upper bound, this gives equality and completes \eqref{eq:I}.

Formulas \eqref{eq:II}--\eqref{eq:IV} follow by the same argument with the symmetric gadget
\[
(\delta r^{-1})^{d-1}\ \delta\ (r\delta)^{d-1}
\]
and the appropriate initial and final shifts as in the source file.
\end{proof}

\section{The first theorem: distance from a shifted reversal to the identity}

Define
\[
s := [n\ n-1\ \cdots\ 1]\in S_n,\qquad r := [2\ 3\ \cdots\ n\ 1]\in S_n.
\]
Note $sr^{n-i}$ is the reversal $s$ shifted (cyclically) so that its first entry is $s_{n-i+1}$.

\begin{theorem}\label{thm:main}
Let $n\ge 4$. For every $i\in\{1,2,\dots,n\}$ the following formula holds:
\[
\dist(sr^{n-i},\id)
= \cnt(\cnt-1)-1 + \fnt(\fnt-1)-1 \;+\;
\begin{cases}
\fnt + 1, & i=1,\\[4pt]
\fnt - i + 4, & 2 \le i \le \fnt+2,\\[4pt]
i - \cnt, & \cnt+2 \le i \le n.
\end{cases}
\]
\end{theorem}

\begin{proof}
We present the proof as an explicit construction whose length is then shown minimal among the two canonical constructions.
All steps are written so they can be converted into a Lean proof by splitting into lemmas about the specific words used.

\subsection*{Step 1: Two-block strategy}
Set $a:=\fnt$ and $b:=\cnt$ so that $a+b=n$ and $b\in\{a,a+1\}$.

Fix an integer $m$ (to be chosen among two special values below) and consider the permutation $sr^{-m}$.
We split its one-line sequence into two consecutive blocks of lengths $a$ and $b$:
\[
\textup{F}_1 := \text{the first $a$ entries of } sr^{-m},\qquad
\textup{F}_2 := \text{the remaining $b$ entries of } sr^{-m}.
\]
Reversing $\textup{F}_1$ and $\textup{F}_2$ (each as a prefix reversal after suitable shifts) transforms $sr^{-m}$ into a pure shift $r^t$
for an explicitly computable exponent $t$.
This is exactly the content of the source argument; here we record it in the form needed for automation.

\subsection*{Step 2: Reverse $\textup{F}_1$ at optimal cost}
Apply Lemma~\ref{lemma:reversal} with $j=a$ to reverse the block $\textup{F}_1$.
Using one of the four equalities (I)--(IV), and choosing $m$ so that no extra preparatory shifts are needed,
we obtain a word of length
\[
a(a-1)-1
\]
that performs the $\textup{F}_1$-reversal and ends at a permutation whose first entries lie in $\textup{F}_2$.

\subsection*{Step 3: Two shifts to expose $\textup{F}_2$}
After the $\textup{F}_1$-reversal, we perform exactly two shifts by $r$ to ensure the second block $\textup{F}_2$ is positioned
so that Lemma~\ref{lemma:reversal} (with $j=b$) applies without extra shifts.
This contributes $2$ to the word length.

\subsection*{Step 4: Reverse $\textup{F}_2$ at optimal cost}
Apply Lemma~\ref{lemma:reversal} with $j=b$ to reverse $\textup{F}_2$.
This contributes
\[
b(b-1)-1
\]
to the length.
After this second reversal, the resulting permutation is a pure shift $r^t$.

\subsection*{Step 5: The two canonical choices of $m$ and the resulting exponents}
The source proof isolates two choices of $m$ that achieve the ``2-shift'' bridge in Step 3:
\[
m_1 := \cfntt-1,\qquad m_2 := \ffntt-1.
\]
For these choices, the two-stage reversal yields the exponents
\[
t_1 := (\cfntt-1)+(\fcntt-1)=\cfntt+\fcntt-2,\qquad
t_2 := (\ffntt-1)+(\ccntt-1)=\ffntt+\ccntt-2.
\]
Thus, for each of the two constructions, we obtain an explicit word sending $sr^{-m_k}$ to $r^{t_k}$
of total length
\[
\bigl(a(a-1)-1\bigr) + 2 + \bigl(b(b-1)-1\bigr).
\]

\subsection*{Step 6: Transfer to $sr^{n-i}$ and add the final Lee distance}
We now substitute $sr^{n-i}$ in place of $sr^{-m}$.
Using $s^2=\id$, we have for each $k\in\{1,2\}$:
\[
sr^{n-i}\cdot \bigl( sr^{t_k}\bigr) = r^{t_k-n+i}.
\]
Hence by Lemma~\ref{lem:lee},
\[
\dist\bigl(sr^{n-i},\id\bigr)
\le \bigl(a(a-1)-1\bigr) + 2 + \bigl(b(b-1)-1\bigr)
+ \min\bigl(|t_k-n+i|,\ n-|t_k-n+i|\bigr).
\]
Taking the better of the two canonical constructions gives
\[
\dist\bigl(sr^{n-i},\id\bigr)
\le \bigl(a(a-1)-1\bigr) + \bigl(b(b-1)-1\bigr)
+ 2 + \min\Bigl(\mathrm{Lee}(t_1-n+i),\ \mathrm{Lee}(t_2-n+i)\Bigr),
\]
where $\mathrm{Lee}(x):=\min(|x|,n-|x|)$.

\subsection*{Step 7: Simplify the minimum to the stated piecewise formula}
A straightforward (but case-heavy) arithmetic check on floors/ceils shows:
\[
2 + \min\Bigl(\mathrm{Lee}(t_1-n+i),\ \mathrm{Lee}(t_2-n+i)\Bigr)
=
\begin{cases}
a+1,& i=1,\\[2pt]
a-i+4,& 2\le i\le a+2,\\[2pt]
i-b,& b+2\le i\le n,
\end{cases}
\]
with $a=\fnt$ and $b=\cnt$.
(For automation, split into the two parity cases $n=2q$ and $n=2q+1$, reduce $t_1,t_2$ to explicit integers,
and then evaluate the Lee minimum by elementary inequalities.)

Combining with $a=\fnt$, $b=\cnt$ yields exactly the claimed expression.
\end{proof}

\section{The second theorem: the diameter lower bound}

\begin{theorem}\label{thm:diam}
A lower bound for the diameter of the Cayley graph of $S_n$ generated by $(12)$, $(12\cdots n)$ and $(1n\cdots 2)$
(equivalently $\delta,r,r^{-1}$) is $\frac{n(n-1)}{2}$.
\end{theorem}

\begin{proof}
By definition, $\diam(\Gamma)\ge \dist(\pi,\xi)$ for every pair $\pi,\xi\in S_n$.
In particular,
\[
\diam(\Gamma)\ \ge\ \dist(sr^{n-2},\id).
\]
Apply Theorem~\ref{thm:main} with $i=2$.
A direct simplification of its piecewise expression at $i=2$ gives
\[
\dist(sr^{n-2},\id)=\frac{n(n-1)}{2}.
\]
Therefore $\diam(\Gamma)\ge \frac{n(n-1)}{2}$.
\end{proof}

\end{document}
